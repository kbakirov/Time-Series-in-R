\PassOptionsToPackage{unicode=true}{hyperref} % options for packages loaded elsewhere
\PassOptionsToPackage{hyphens}{url}
%
\documentclass[]{article}
\usepackage{lmodern}
\usepackage{amssymb,amsmath}
\usepackage{ifxetex,ifluatex}
\usepackage{fixltx2e} % provides \textsubscript
\ifnum 0\ifxetex 1\fi\ifluatex 1\fi=0 % if pdftex
  \usepackage[T1]{fontenc}
  \usepackage[utf8]{inputenc}
  \usepackage{textcomp} % provides euro and other symbols
\else % if luatex or xelatex
  \usepackage{unicode-math}
  \defaultfontfeatures{Ligatures=TeX,Scale=MatchLowercase}
\fi
% use upquote if available, for straight quotes in verbatim environments
\IfFileExists{upquote.sty}{\usepackage{upquote}}{}
% use microtype if available
\IfFileExists{microtype.sty}{%
\usepackage[]{microtype}
\UseMicrotypeSet[protrusion]{basicmath} % disable protrusion for tt fonts
}{}
\IfFileExists{parskip.sty}{%
\usepackage{parskip}
}{% else
\setlength{\parindent}{0pt}
\setlength{\parskip}{6pt plus 2pt minus 1pt}
}
\usepackage{hyperref}
\hypersetup{
            pdftitle={Forecast.R},
            pdfauthor={Kuanysh},
            pdfborder={0 0 0},
            breaklinks=true}
\urlstyle{same}  % don't use monospace font for urls
\usepackage[margin=1in]{geometry}
\usepackage{color}
\usepackage{fancyvrb}
\newcommand{\VerbBar}{|}
\newcommand{\VERB}{\Verb[commandchars=\\\{\}]}
\DefineVerbatimEnvironment{Highlighting}{Verbatim}{commandchars=\\\{\}}
% Add ',fontsize=\small' for more characters per line
\usepackage{framed}
\definecolor{shadecolor}{RGB}{248,248,248}
\newenvironment{Shaded}{\begin{snugshade}}{\end{snugshade}}
\newcommand{\AlertTok}[1]{\textcolor[rgb]{0.94,0.16,0.16}{#1}}
\newcommand{\AnnotationTok}[1]{\textcolor[rgb]{0.56,0.35,0.01}{\textbf{\textit{#1}}}}
\newcommand{\AttributeTok}[1]{\textcolor[rgb]{0.77,0.63,0.00}{#1}}
\newcommand{\BaseNTok}[1]{\textcolor[rgb]{0.00,0.00,0.81}{#1}}
\newcommand{\BuiltInTok}[1]{#1}
\newcommand{\CharTok}[1]{\textcolor[rgb]{0.31,0.60,0.02}{#1}}
\newcommand{\CommentTok}[1]{\textcolor[rgb]{0.56,0.35,0.01}{\textit{#1}}}
\newcommand{\CommentVarTok}[1]{\textcolor[rgb]{0.56,0.35,0.01}{\textbf{\textit{#1}}}}
\newcommand{\ConstantTok}[1]{\textcolor[rgb]{0.00,0.00,0.00}{#1}}
\newcommand{\ControlFlowTok}[1]{\textcolor[rgb]{0.13,0.29,0.53}{\textbf{#1}}}
\newcommand{\DataTypeTok}[1]{\textcolor[rgb]{0.13,0.29,0.53}{#1}}
\newcommand{\DecValTok}[1]{\textcolor[rgb]{0.00,0.00,0.81}{#1}}
\newcommand{\DocumentationTok}[1]{\textcolor[rgb]{0.56,0.35,0.01}{\textbf{\textit{#1}}}}
\newcommand{\ErrorTok}[1]{\textcolor[rgb]{0.64,0.00,0.00}{\textbf{#1}}}
\newcommand{\ExtensionTok}[1]{#1}
\newcommand{\FloatTok}[1]{\textcolor[rgb]{0.00,0.00,0.81}{#1}}
\newcommand{\FunctionTok}[1]{\textcolor[rgb]{0.00,0.00,0.00}{#1}}
\newcommand{\ImportTok}[1]{#1}
\newcommand{\InformationTok}[1]{\textcolor[rgb]{0.56,0.35,0.01}{\textbf{\textit{#1}}}}
\newcommand{\KeywordTok}[1]{\textcolor[rgb]{0.13,0.29,0.53}{\textbf{#1}}}
\newcommand{\NormalTok}[1]{#1}
\newcommand{\OperatorTok}[1]{\textcolor[rgb]{0.81,0.36,0.00}{\textbf{#1}}}
\newcommand{\OtherTok}[1]{\textcolor[rgb]{0.56,0.35,0.01}{#1}}
\newcommand{\PreprocessorTok}[1]{\textcolor[rgb]{0.56,0.35,0.01}{\textit{#1}}}
\newcommand{\RegionMarkerTok}[1]{#1}
\newcommand{\SpecialCharTok}[1]{\textcolor[rgb]{0.00,0.00,0.00}{#1}}
\newcommand{\SpecialStringTok}[1]{\textcolor[rgb]{0.31,0.60,0.02}{#1}}
\newcommand{\StringTok}[1]{\textcolor[rgb]{0.31,0.60,0.02}{#1}}
\newcommand{\VariableTok}[1]{\textcolor[rgb]{0.00,0.00,0.00}{#1}}
\newcommand{\VerbatimStringTok}[1]{\textcolor[rgb]{0.31,0.60,0.02}{#1}}
\newcommand{\WarningTok}[1]{\textcolor[rgb]{0.56,0.35,0.01}{\textbf{\textit{#1}}}}
\usepackage{graphicx,grffile}
\makeatletter
\def\maxwidth{\ifdim\Gin@nat@width>\linewidth\linewidth\else\Gin@nat@width\fi}
\def\maxheight{\ifdim\Gin@nat@height>\textheight\textheight\else\Gin@nat@height\fi}
\makeatother
% Scale images if necessary, so that they will not overflow the page
% margins by default, and it is still possible to overwrite the defaults
% using explicit options in \includegraphics[width, height, ...]{}
\setkeys{Gin}{width=\maxwidth,height=\maxheight,keepaspectratio}
\setlength{\emergencystretch}{3em}  % prevent overfull lines
\providecommand{\tightlist}{%
  \setlength{\itemsep}{0pt}\setlength{\parskip}{0pt}}
\setcounter{secnumdepth}{0}
% Redefines (sub)paragraphs to behave more like sections
\ifx\paragraph\undefined\else
\let\oldparagraph\paragraph
\renewcommand{\paragraph}[1]{\oldparagraph{#1}\mbox{}}
\fi
\ifx\subparagraph\undefined\else
\let\oldsubparagraph\subparagraph
\renewcommand{\subparagraph}[1]{\oldsubparagraph{#1}\mbox{}}
\fi

% set default figure placement to htbp
\makeatletter
\def\fps@figure{htbp}
\makeatother


\title{Forecast.R}
\author{Kuanysh}
\date{2020-03-03}

\begin{document}
\maketitle

\begin{Shaded}
\begin{Highlighting}[]
\CommentTok{## Importing packages}
\KeywordTok{library}\NormalTok{(readr)}
\KeywordTok{library}\NormalTok{(plyr)}
\KeywordTok{library}\NormalTok{(astsa)}
\KeywordTok{library}\NormalTok{(ggplot2)}
\KeywordTok{library}\NormalTok{(xts)}
\KeywordTok{library}\NormalTok{(forecast)}
\KeywordTok{library}\NormalTok{(fGarch)}
\KeywordTok{library}\NormalTok{(fpp)}
\KeywordTok{library}\NormalTok{(tidyverse)}
\KeywordTok{library}\NormalTok{(Metrics)}

\KeywordTok{getwd}\NormalTok{()}
\end{Highlighting}
\end{Shaded}

\begin{verbatim}
## [1] "C:/Users/Kuanysh/Documents/GitHub/Time-Series-in-R/Exam"
\end{verbatim}

\begin{Shaded}
\begin{Highlighting}[]
\KeywordTok{setwd}\NormalTok{(}\StringTok{"C:/Users/Kuanysh/Documents/GitHub/Time-Series-in-R/Exam"}\NormalTok{)}
\NormalTok{data <-}\StringTok{ }\KeywordTok{read.csv}\NormalTok{(}\StringTok{"C:/Users/Kuanysh/Documents/GitHub/Time-Series-in-R/Exam/Elec-train.csv"}\NormalTok{)}

\CommentTok{#Rename}
\KeywordTok{names}\NormalTok{(data)[}\DecValTok{2}\NormalTok{]<-}\StringTok{"Power"}
\KeywordTok{names}\NormalTok{(data)[}\DecValTok{3}\NormalTok{]<-}\StringTok{"Temperature"}

\CommentTok{#Date format}
\NormalTok{data}\OperatorTok{$}\NormalTok{Timestamp <-}\StringTok{ }\KeywordTok{as.POSIXct}\NormalTok{(data}\OperatorTok{$}\NormalTok{Timestamp, }\DataTypeTok{format =}\StringTok{"%m/%d/%Y %H:%M"}\NormalTok{, }\DataTypeTok{tz =} \StringTok{"GMT"}\NormalTok{)}
\NormalTok{power.ts <-}\StringTok{ }\KeywordTok{ts}\NormalTok{(data}\OperatorTok{$}\NormalTok{Power, }\DataTypeTok{frequency =} \DecValTok{96}\NormalTok{)}
\NormalTok{data}\OperatorTok{$}\NormalTok{time <-}\StringTok{ }\KeywordTok{as.numeric}\NormalTok{(}\KeywordTok{time}\NormalTok{(power.ts))}
\NormalTok{temperature.ts <-}\StringTok{ }\KeywordTok{ts}\NormalTok{(data}\OperatorTok{$}\NormalTok{Temperature, }\DataTypeTok{frequency =} \DecValTok{96}\NormalTok{)}

\CommentTok{#Plot}
\KeywordTok{autoplot}\NormalTok{(power.ts)}\OperatorTok{+}
\StringTok{  }\KeywordTok{ggtitle}\NormalTok{(}\StringTok{'Power consumption per day'}\NormalTok{)}\OperatorTok{+}
\StringTok{  }\KeywordTok{xlab}\NormalTok{(}\StringTok{'Days'}\NormalTok{)}\OperatorTok{+}
\StringTok{  }\KeywordTok{ylab}\NormalTok{(}\StringTok{'Power'}\NormalTok{)}
\end{Highlighting}
\end{Shaded}

\includegraphics{Forecast_files/figure-latex/unnamed-chunk-1-1.pdf}

\begin{Shaded}
\begin{Highlighting}[]
\KeywordTok{autoplot}\NormalTok{(temperature.ts)}\OperatorTok{+}
\StringTok{  }\KeywordTok{ggtitle}\NormalTok{(}\StringTok{'Temperature per day'}\NormalTok{)}\OperatorTok{+}
\StringTok{  }\KeywordTok{xlab}\NormalTok{(}\StringTok{'Days'}\NormalTok{)}\OperatorTok{+}
\StringTok{  }\KeywordTok{ylab}\NormalTok{(}\StringTok{'Temperature'}\NormalTok{)}
\end{Highlighting}
\end{Shaded}

\includegraphics{Forecast_files/figure-latex/unnamed-chunk-1-2.pdf}

\begin{Shaded}
\begin{Highlighting}[]
\CommentTok{#use decompose function to check the seasonal trend and trend for itself}
\CommentTok{#plot(decompose(power.ts))}

\CommentTok{#Splitting data}
\NormalTok{nvaldays <-}\StringTok{ }\DecValTok{3}

\NormalTok{test.power <-}\StringTok{ }\KeywordTok{tail}\NormalTok{(data}\OperatorTok{$}\NormalTok{Power, }\DecValTok{96}\NormalTok{)}
\NormalTok{full.train.power <-}\StringTok{ }\KeywordTok{head}\NormalTok{(data}\OperatorTok{$}\NormalTok{Power, }\DecValTok{-96}\NormalTok{)}
\NormalTok{train.power <-}\StringTok{ }\KeywordTok{head}\NormalTok{(full.train.power, }\OperatorTok{-}\NormalTok{nvaldays}\OperatorTok{*}\DecValTok{96}\NormalTok{)}
\NormalTok{val.power <-}\StringTok{ }\KeywordTok{tail}\NormalTok{(train.power,nvaldays}\OperatorTok{*}\DecValTok{96}\NormalTok{)}


\NormalTok{val.time <-}\StringTok{ }\KeywordTok{tail}\NormalTok{(}\KeywordTok{as.numeric}\NormalTok{(}\KeywordTok{time}\NormalTok{(}\KeywordTok{ts}\NormalTok{(full.train.power, }\DataTypeTok{frequency =} \DecValTok{96}\NormalTok{))), nvaldays}\OperatorTok{*}\DecValTok{96}\NormalTok{)}

\KeywordTok{plot}\NormalTok{(}\KeywordTok{ts}\NormalTok{(train.power, }\DataTypeTok{frequency =} \DecValTok{96}\NormalTok{),}\DataTypeTok{xlim=}\KeywordTok{c}\NormalTok{(}\DecValTok{0}\NormalTok{,}\DecValTok{50}\NormalTok{))}
\KeywordTok{par}\NormalTok{(}\DataTypeTok{new=}\OtherTok{TRUE}\NormalTok{)}
\KeywordTok{lines}\NormalTok{(val.time, val.power, }\DataTypeTok{col=}\StringTok{"red"}\NormalTok{, }\DataTypeTok{xlim=}\KeywordTok{c}\NormalTok{(}\DecValTok{0}\NormalTok{,}\DecValTok{50}\NormalTok{))}
\end{Highlighting}
\end{Shaded}

\includegraphics{Forecast_files/figure-latex/unnamed-chunk-1-3.pdf}

\begin{Shaded}
\begin{Highlighting}[]
\CommentTok{#Forecast}

\CommentTok{# simple ES with only alpha}
\NormalTok{Power<-}\KeywordTok{ts}\NormalTok{(val.power, }\DataTypeTok{frequency =} \DecValTok{96}\NormalTok{)}
\KeywordTok{plot}\NormalTok{(Power,}\DataTypeTok{col=}\StringTok{"red"}\NormalTok{)}
\NormalTok{SES=}\KeywordTok{HoltWinters}\NormalTok{(Power,}\DataTypeTok{alpha=}\OtherTok{NULL}\NormalTok{,}\DataTypeTok{beta=}\OtherTok{FALSE}\NormalTok{,}\DataTypeTok{gamma=}\OtherTok{FALSE}\NormalTok{)}
\NormalTok{p1<-}\KeywordTok{predict}\NormalTok{(SES,}\DataTypeTok{n.ahead=}\NormalTok{nvaldays}\OperatorTok{*}\DecValTok{96}\NormalTok{)}
\KeywordTok{par}\NormalTok{(}\DataTypeTok{new=}\OtherTok{TRUE}\NormalTok{)}
\KeywordTok{plot}\NormalTok{(}\KeywordTok{ts}\NormalTok{(}\KeywordTok{as.numeric}\NormalTok{(p1),}\DataTypeTok{frequency =} \DecValTok{96}\NormalTok{),}\DataTypeTok{col=}\DecValTok{3}\NormalTok{,}\DataTypeTok{ann=}\OtherTok{FALSE}\NormalTok{,}\DataTypeTok{axes=}\OtherTok{FALSE}\NormalTok{)}
\KeywordTok{rmse}\NormalTok{(val.power, }\KeywordTok{as.numeric}\NormalTok{(p1))}
\end{Highlighting}
\end{Shaded}

\begin{verbatim}
## [1] 88.22327
\end{verbatim}

\begin{Shaded}
\begin{Highlighting}[]
\CommentTok{# full ES with alpha beta gamma}
\CommentTok{#plot(ts(val.power, frequency = 96),col="red")}
\NormalTok{SES=}\KeywordTok{HoltWinters}\NormalTok{(Power,}\DataTypeTok{alpha=}\OtherTok{NULL}\NormalTok{,}\DataTypeTok{beta=}\OtherTok{NULL}\NormalTok{,}\DataTypeTok{gamma=}\OtherTok{NULL}\NormalTok{)}
\NormalTok{p1<-}\KeywordTok{predict}\NormalTok{(SES,}\DataTypeTok{n.ahead=}\NormalTok{nvaldays}\OperatorTok{*}\DecValTok{96}\NormalTok{)}
\KeywordTok{par}\NormalTok{(}\DataTypeTok{new=}\OtherTok{TRUE}\NormalTok{)}
\KeywordTok{plot}\NormalTok{(}\KeywordTok{ts}\NormalTok{(}\KeywordTok{as.numeric}\NormalTok{(p1),}\DataTypeTok{frequency =} \DecValTok{96}\NormalTok{),}\DataTypeTok{col=}\DecValTok{6}\NormalTok{,}\DataTypeTok{ann=}\OtherTok{FALSE}\NormalTok{,}\DataTypeTok{axes=}\OtherTok{FALSE}\NormalTok{)}
\end{Highlighting}
\end{Shaded}

\includegraphics{Forecast_files/figure-latex/unnamed-chunk-1-4.pdf}

\begin{Shaded}
\begin{Highlighting}[]
\KeywordTok{rmse}\NormalTok{(val.power, }\KeywordTok{as.numeric}\NormalTok{(p1))}
\end{Highlighting}
\end{Shaded}

\begin{verbatim}
## [1] 14.03703
\end{verbatim}

\end{document}
